%% -*- mode: latex -*-
%% Time-stamp: "2024-11-06 09:04:52 (ywatanabe)"
%% File: ./torchPAC/paper/manuscript/src/results.tex

\section{Results}

We developed a novel computational framework for trainable phase-amplitude coupling (PAC) analysis implemented in PyTorch. The framework enables end-to-end optimization of PAC parameters through gradient descent while maintaining neurophysiological interpretability.

\subsection{Schematic Overview}
\subsection{Data Preparation}
We validated our framework using two types of datasets. First, we generated synthetic data with known ground truth coupling between low-frequency phase (4-8 Hz) and high-frequency amplitude (80-150 Hz) components \hlref{fig:00_synthetic_raw}. The synthetic dataset included 1000 trials with varying coupling strengths and phase preferences. Second, we analyzed EEG recordings from ... during ..., focusing on theta-gamma coupling \hlref{fig:00_phisiiological_raw}.

\subsection{Phase-Amplitude Coupling}
The PAC computation follows established methods while introducing trainable parameters \hlref{fig:00_designe}. The signal first undergoes bandpass filtering using finite impulse response (FIR) filters with learnable cut-off frequencies \hlref{fig:00_bandpass}. Hilbert transformation extracts instantaneous phase and amplitude from the filtered signals \hlref{fig:00_hilbert}. The modulation index quantifies the coupling strength between the phase of slower oscillations and the amplitude of faster oscillations \hlref{fig:modulation_index}.

\subsection{PAC Value Confirmation with an Existing Package}
PAC value comparison \hlref{fig:00_pac_value_comparision}.

\subsection{Speed Comparison with an Existing Package}
batch size (Figure~\hlref{fig:01_batch_size})\\
chunk size (Figure~\hlref{fig:02_chunk_size})\\
number of channels (Figure~\hlref{fig:03_n_chs})\\
duration (Figure~\hlref{fig:04_t_sec})\\
sampling frequency (Figure~\hlref{fig:05_fs})\\
number of frequency bands for phase (Figure~\hlref{fig:07_pha_n_bands})\\
number of frequency bands for amplitude (Figure~\hlref{fig:08_n_perm})\\

\subsection{Trainable Phase-Amplitude Coupling}
Another key innovation is making PAC parameters fully differenciable, for being trainable through backpropagation algorithms. Specifically, we implemented: (i) Learnable filter parameters for optimal frequency band selection (ii), (ii) differenciable hilbert transformation, (iii) adaptable phase-amplitude binning for modulation index calculation. To demonstrate, we trained a model with trainable PAC module for a classification task distinguishing between coupled and uncoupled oscillations. The framework achieved 95\% classification accuracy on synthetic data and successfully identified physiological theta-gamma coupling patterns.
