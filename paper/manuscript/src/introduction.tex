%% -*- mode: latex -*-
%% Time-stamp: "2024-11-06 09:10:28 (ywatanabe)"
%% File: ./torchPAC/paper/manuscript/src/introduction.tex

\section{Introduction}
[START of 1. Opening Statement] Pattern recognition and machine learning have revolutionized data analysis across scientific disciplines, from computer vision to natural language processing, through their ability to extract meaningful patterns from complex datasets. [END of 1. Opening Statement] [START of 2. Importance of the Field] These computational approaches have become particularly crucial in neuroscience, where understanding intricate neural dynamics requires processing and analyzing vast amounts of high-dimensional data. [END of 2. Importance of the Field]

[START of 3. Existing Knowledge and Gaps] Neural oscillations represent a fundamental mechanism for information processing and communication in the brain. Phase-amplitude coupling (PAC), which quantifies the interaction between different frequency components of neural signals, has emerged as a valuable biomarker for neural functioning. In the hippocampus, high-frequency ripples (150-250 Hz) are temporally coupled with low-frequency sharp waves (0.5-2 Hz), a phenomenon critical for memory consolidation and spatial navigation. Similarly, phase precession, where neuronal firing systematically shifts relative to theta rhythms (4-8 Hz), serves as a neural code for working memory and spatial information. Additionally, alterations in PAC patterns characterize various pathological conditions, including epilepsy, where abnormal coupling between different frequency bands often precedes seizure onset. [END of 3. Existing Knowledge and Gaps]

[START of 4. Limitations in Previous Works] Despite its biological significance, PAC analysis faces computational challenges that limit its practical applications. Traditional PAC calculation methods require multiple sequential processing steps: bandpass filtering, Hilbert transformation, modulation index computation, and surrogate data comparison. These computationally intensive procedures become particularly problematic when analyzing large-scale neural recordings or implementing real-time applications. Moreover, current software implementations often lack optimization for modern hardware architectures, resulting in processing bottlenecks that constrain both research scope and clinical applications. [END of 4. Limitations in Previous Works]

[START of 5. Research Question or Hypothesis] We hypothesized that leveraging Graphics Processing Unit (GPU) acceleration could dramatically enhance PAC computation efficiency while maintaining accuracy, enabling both large-scale analyses and real-time applications. Additionally, we proposed that integrating PAC calculation into deep learning frameworks would facilitate end-to-end optimization of frequency band parameters. [END of 5. Research Question or Hypothesis]

[START of 6. Approach and Methods] Our approach implements a parallel computing framework utilizing General-Purpose GPU (GPGPU) architecture through PyTorch, a popular deep learning library. This implementation capitalizes on the independent nature of PAC computation steps, enabling simultaneous processing of multiple data streams. We optimized each computational stage for parallel execution, from filtering to statistical analysis, addressing the bottlenecks inherent in traditional CPU-based approaches. Furthermore, we designed the framework to be compatible with automatic differentiation, allowing for gradient-based optimization of frequency band parameters. [END of 6. Approach and Methods]

[START of 7. Overview of Results] Our GPU-accelerated implementation demonstrated a 100-fold speedup compared to conventional CPU-based methods while maintaining mathematical equivalence with established PAC metrics. The framework successfully processed terabyte-scale neural recordings and enabled real-time PAC computation. Moreover, our trainable PAC module revealed optimal frequency bands for specific neural processes through data-driven optimization. [END of 7. Overview of Results]

[START of 8. Significance and Implications] This advancement in PAC computation efficiency opens new possibilities for analyzing larger datasets and implementing real-time neural signal processing, potentially enabling novel applications in brain-computer interfaces and clinical monitoring systems. As an open-source framework, our implementation contributes to the broader neuroscience community's efforts to understand complex neural dynamics and develop more effective therapeutic interventions. [END of 8. Significance and Implications]

\label{sec:introduction}
