%% -*- mode: latex -*-
%% Time-stamp: "2024-11-04 14:48:19 (ywatanabe)"
%% File: ./torchPAC/paper/manuscript/src/methods.tex

\section{Methods}
\subsection{Synthetic Data}
We utilized synthetic data for profiling computational speed and accuracy.

\subsection{Physiological Data}
Additionally, we verified our method using physiological recordings from [fixme ->] XXX [<- fixme] for event-related analyses.


\subsection{Implementation of GPU-accelerated PAC}
To enable seamless integration with artificial intelligence (AI) training frameworks, we developed a graphics processing unit (GPU)-accelerated phase-amplitude coupling (PAC) implementation using PyTorch as the computational foundation. The implementation comprises three primary components: bandpass filtering, Hilbert transformation, and mutual information index calculations, which are modularly integrated into a unified PAC class and function. This implementation is publicly available in the mngs package, an open-source Python toolbox (https://github.com/ywata1989/mngs/dsp).

\indent GPU-accelerated PAC calculation can be executed with three lines of code:
\begin{verbatim}
import mngs
signal, _time, fs = mngs.dsp.demo_sig()
pac, freqs_pha, freqs_amp = mngs.dsp.pac(signal, fs, batch_size=1, batch_size_ch=1, n_perm=20)
\end{verbatim}
where \texttt{signal} represents the input time series data ($\mathbb{R}^{n_\text{samples} \times n_\text{channels} \times n_\text{sequence}}$), \texttt{\_time} contains the corresponding time points, \texttt{fs} specifies the sampling frequency in Hz, \texttt{batch\_size} defines the number of temporal segments processed simultaneously, \texttt{batch\_size\_ch} specifies the number of channels processed in parallel, \texttt{n\_perm} indicates the number of permutations for surrogate testing, \texttt{pac} returns the calculated PAC values, and \texttt{freqs\_pha} and \texttt{freqs\_amp} represent the frequency bands for phase and amplitude components, respectively.

\subsection{Machine Specification}
All computations were performed on a workstation running Rocky Linux 9.4 with an AMD Ryzen 9 7950X 16-core/32-thread CPU (maximum frequency: 5.88 GHz) and 61.7 GiB of RAM. GPU acceleration was implemented using an NVIDIA GeForce RTX 4090 with CUDA 12.6.20. Our implementation utilized PyTorch [fixme ->] version X.X.X [<- fixme] and was tested on both CPU and GPU configurations.

\subsection{Calculation Quality}
Mean squared error (MSE) was employed to measure calculation differences between our implementation and an existing PAC calculation package, TensorPAC.


\subsection{Speed Comparison}
Performance benchmarking was conducted using a baseline data chunk of dimensions $(n_\text{samples}, n_\text{channels}, n_\text{sequence}) = (4, 19, 2^8)$. Each condition was measured three times with the following parameters:

- Batch size: $2^3, 2^4, 2^5, 2^6$
- Number of channels: $2^3, 2^4, 2^5, 2^6$  
- Number of segments: $2^0, 2^1, 2^2, 2^3, 2^4$
- Time duration: $2^0, 2^1, 2^2, 2^3$ seconds
- Sampling rate: $2^9, 2^{10}$ Hz
- Phase frequency bands: $10, 30, 50, 70, 10^2$
- Amplitude frequency bands: $10, 30, 50, 70, 10^2$
- Number of permutations: $2^0, 2^1, 2^2$
- Chunk size: $2^0, 2^1, 2^2, 2^3$
- FP16 precision: enabled, disabled
- Gradient calculation: enabled, disabled
- In-place operations: enabled, disabled
- Model trainability: enabled, disabled
- Computing device: CPU, GPU (CUDA)
- Multi-threading: enabled, disabled
- Number of calculations: $2^0, 2^1, 2^2, 2^3$

Computation times were compared between TensorPAC and our mngs package implementation across all parameter combinations to assess relative performance advantages.



\subsection{Statistical Evaluation}
Both the Brunner--Munzel test and the Kruskal--Wallis test were executed using the SciPy package in Python \cite{virtanen_scipy_2020}. Correlational analysis was conducted by determining the rank of the observed correlation coefficient within its associated set-size-shuffled surrogate using a customized Python script. The bootstrap test was implemented with an in-house Python script.
\label{sec:methods}

