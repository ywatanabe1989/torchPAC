\section{Introduction}
PAC is considered to reflect neural phenomenon.

However, calculation speed is quite slow and thus researchers struggle to proceed the revenue.

Also, another problem lies in how to find optimal bands for phase and amplitude under an aim and tasks.

To clear these problems, here we introduce torchPAC, a pytorch-based PAC calculation module with learnable filters, which are fully differentiable and thus trainable with the backpropagation.

% \\
% \indent
% Current evidence suggests that a transient, synchronized oscillation, called sharp-wave ripples (SWRs) \cite{buzsaki_hippocampal_2015}, is associated with several cognitive functions. These include memory replay \cite{wilson_reactivation_1994} \cite{nadasdy_replay_1999} \cite{lee_memory_2002} \cite{diba_forward_2007} \cite{davidson_hippocampal_2009}, memory consolidation \cite{girardeau_selective_2009} \cite{ego-stengel_disruption_2010} \cite{fernandez-ruiz_long-duration_2019} \cite{kim_corticalhippocampal_2022}, memory recall \cite{wu_hippocampal_2017} \cite{norman_hippocampal_2019} \cite{norman_hippocampal_2021}, and neural plasticity \cite{behrens_induction_2005} \cite{norimoto_hippocampal_2018}. These associations suggest that SWR may be a fundamental computational manifestation of hippocampal processing, contributing to working memory performance as well.
% \\
% \indent
% Recent studies have found that low-dimensional representations in hippocampal neurons can explain WM task performances. Specifically, the firing patterns of place cells \cite{okeefe_hippocampus_1971} \cite{okeefe_place_1976} \cite{ekstrom_cellular_2003} \cite{kjelstrup_finite_2008} \cite{harvey_intracellular_2009}, found in the hippocampus, have been identified within dynamic, nonlinear three-dimensional hyperbolic spaces in rats \cite{zhang_hippocampal_2022}. Additionally, grid cells in the entorhinal cortex (EC), which is the main pathway to the hippocampus \cite{naber_reciprocal_2001} \cite{van_strien_anatomy_2009} \cite{strange_functional_2014}, exhibited a toroidal geometry during exploration in rats \cite{gardner_toroidal_2022}.
% \\
% \indent
% However, these studies primarily focus on spatial navigation in rodents, which poses limitations. To illustrate, the temporal resolution of navigation tasks is inadequate as the timing of memory acquisition and recall is not clearly delineated. Consequently, there is a relative paucity of research on the impact of SWRs on WM performance \cite{jadhav_awake_2012}. Further, the presence of noise in signals recorded during rodent movement complicates the detection of SWRs \cite{Watanabe_2021}. Therefore, to clarify the relationship between SWRs and WM tasks, research with human subjects is necessary.
% \\
% \indent
% Considering these factors, this study investigates the hypothesis that hippocampal neurons exhibit unique neural trajectories (NTs) in low-dimensional space, particularly during SWR periods, in response to WM tasks in humans. To test this hypothesis, we employed a dataset of patients performing an eight-second Sternberg task (1 s for fixation, 2 s for encoding, 3 s for maintenance, and 2 s for retrieval) with high temporal resolution. Intracranial electroencephalography (iEEG) signals within the medial temporal lobe (MTL) were recorded for these patients \cite{boran_dataset_2020}. To investigate low-dimensional NTs, we utilized Gaussian-process factor analysis (GPFA), an established method for analyzing neural population dynamics \cite{yu_gaussian-process_2009}.
\label{sec:introduction}