\begin{abstract}
  \pdfbookmark[1]{Abstract}{abstract}
Working memory (WM) is crucial for a multitude of cognitive functions, but its underlying neural mechanisms at play remain to be fully understood. Interestingly, the hippocampus and sharp-wave ripples (SWRs)--- transient and synchronous oscillation observed in the hippocampus --- garner recognition for their significance in memory consolidation and retrieval, albeit their correlation with WM tasks remains to be elucidated. Here we show that during a WM task, multi-unit activity patterns in the hippocampus display distinctive dynamics, especially during SWR periods. This study analyzed intracranial electroencephalography data from the medial temporal lobe (MTL) of nine patients with epilepsy, recorded during an eight-second Sternberg task. Utilizing Gaussian-process factor analysis, we extracted low-dimensional neural representations or trajectories (NT) within MTL regions during the Sternberg WM task. The results revealed significant variations in NT of the hippocampus compared to those of the entorhinal cortex and the amygdala. Additionally, the distance of the NT between the encoding and retrieval phases was dependent on memory load. Crucially, hippocampal NT oscillated during the retrieval phase between encoding and retrieval states in a task-dependent manner. Notably, these fluctuations shifted from encoding to retrieval states during SWR episodes. These findings reaffirm the role of the hippocampus in WM tasks and propose a hypothesis: the hippocampus transitions its functional state from encoding to retrieval during SWRs in WM tasks.
\end{abstract}

